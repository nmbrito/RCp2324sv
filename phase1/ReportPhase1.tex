\documentclass[11pt]{report}
\title{Computer Networks}
\author{Nuno Brito}
%\date{}

%packaging
\usepackage{fullpage}									% Utilizar página inteira
\usepackage{tikz}										% Figuras tikz (construção de vetores, fluxogramas, etc.)
\usepackage{comment}									% Permite criar um segmento de comentários (/begin{comment} [...] /end{comment})
\usepackage{xcolor}										% Colorir texto (esta versão é mais flexível do que a {color})
\usepackage{indentfirst}								% Indenta SEMPRE o primeiro parágrafo (/noindent para não indentar)
\usepackage{graphicx}									% Inserir imagens
\usepackage{placeins}									% Controla o posicionamento de tabelas e figuras logo após o texto (/FloatBarrier)
\usepackage{subcaption}									% Subcaptions para figuras
\usepackage{flafter}									% Qualquer coisa com o posicionamento dos Floats
\usepackage[portuguese]{babel}							% Pacote que descarrega e traduz linguagem global dos campos pré-definidos de LaTeX
\usepackage{multicol}									% Criar texto em colunas (\begin{multicols}{#} [...] \end{multicols}). Usar \raggedcolumns e \columnbreak
\usepackage{titling}									% Centrar, horizontal e verticalmente o título na página (/begin{titling} [...] /end{titling})
\usepackage[linesnumbered,ruled,vlined]{algorithm2e}	% Criação de algoritmos
\usepackage[acronym]{glossaries}						% Permite criação de glossários. Usar \makeglossaries \newglossaryentry{palavra} e \printglossaries
\usepackage{fancyhdr}									% Criação de cabeçalhos e rodapés

%settings
\selectlanguage{portuguese}								% Define linguagem para português
\setlength{\columnsep}{3cm}								% Define espaçamento entre múltiplas colunas
\graphicspath{ {./imagens/} }							% Caminho para o recurso imagens
\renewcommand\maketitlehooka{\null\mbox{}\vfill}		% Comando adicional para centrar o título
\renewcommand\maketitlehookd{\vfill\null}				% Comando adicional 2 para centrar o título
\usetikzlibrary{shapes.geometric, arrows}
\tikzstyle{startstop} = [rectangle, rounded corners, minimum width=3cm, minimum height=1cm,text centered, draw=black, fill=red!30]
\tikzstyle{io} = [trapezium, trapezium left angle=70, trapezium right angle=110, minimum width=3cm, minimum height=1cm, text centered, draw=black, fill=blue!30]
\tikzstyle{process} = [rectangle, minimum width=3cm, minimum height=1cm, text centered, draw=black, fill=orange!30]
\tikzstyle{decision} = [diamond, minimum width=3cm, minimum height=1cm, text centered, draw=black, fill=green!30]
\tikzstyle{arrow} = [thick,->,>=stealth]
\pagestyle{fancy}
\fancyhf{}
\fancyhead[LE, RO]{ISEL}
\fancyhead[RE, LO]{Phase 1 - 25\%}
\fancyfoot[CE, CO]{\leftmark}
\fancyfoot[LE, RO]{\thepage}

\begin{document}

\tableofcontents

\section{Introduction}
Milestones:
\begin{itemize}
    \item Getting apache2 web server running in localhost
    \item Access web server - http://127.0.0.1/
    \item Access web server from other computer
    \item Use wireshark to capture the web access from another host
    \item Compare the HTTP headers sent by the client and the server
    \item Develop a web client
    \item Establish a TCP connection to the server
    \item Request the base webpage
\end{itemize}

Web client requirements:
\begin{itemize}
    \item Usage of HTTP library forbidden
    \item Establish TCP connection using available sockets library - send the HTTP request and receive the HTTP reply
    \item HTTP reply should be presented to the user
    \item - Optional - act to the various HTTP replies
    \item Text-only application
\end{itemize}

\section{Project}
\subsection{Software}
    local server:
        windows 11 x64
        xampp-portable-windows-x64-8.2.12-0-VS16.7z
    client:
        firefox version
        librefox version
        wireshark version

\subsection{Configurations}
    SSLEngine disabled in apache to ease access with http
    <=++figure sslengine off++=>
    <=++figure access from localhost++=>
    <=++figure access from 127.0.0.1++=>
    <=++figure access from anotherhost++=>
    <=++figure wireshark capture from another host++=>

\subsection{Python code}
    """
    Name: Python TCP Client
    Description: Simple TCP client using sockets
    Original code by: Luís Pires
    Source: Chapter 2, slide 104
    
    Commented and adapted by: Nuno Brito
    """
    
    # Import from everything from the socket library
    from socket import *
    
    # Specify servername and port destination
    serverName = ’servername’
    serverPort = 12000
    
    # Socket open and connect
    clientSocket = socket(AF_INET, SOCK_STREAM)
    clientSocket.connect((serverName,serverPort))
    
    # HTTP1.1/GET message
    #   Input manually
    sentence = raw_input(‘Input lowercase sentence:’)
    #   Cycle through a list
    # TODO!!!
    
    # Socket send
    clientSocket.send(sentence.encode())
    
    # Receive and output the response message
    modifiedSentence = clientSocket.recv(1024)
    print (‘From Server:’, modifiedSentence.decode())
    
    # Socket close connection
    clientSocket.close()

\subsection{List of headers}
    client
        meaning
    server
        meaning

\subsection{Objectives}

Step-by-step instructions performed
Xampp install
wireshark install
wireshark settings
wireshark filters

\end{document}
