%% Document template source: LaTeX2e template for FEUP's Project FE-UP
%% Document template author: jlopes@fe.up.pt
%% Template adapted

%% A alterar: <--ALTERAR-->

\documentclass[11pt,a4paper]{report}

%% Macros ----------------------------------------------------------------------
\newcommand{\school}{Instituto Superior de Engenharia de Lisboa}
\newcommand{\degree}{Licenciatura em Engenharia Eletrotécnica, Telecomunicações e Computadores}
\newcommand{\projisel}{Projeto ISEL 2023/24 --- LEETC}
\newcommand{\projtitle}{Computer Networks}
\newcommand{\projsubtitle}{Phase 4 - Deploy Services}
\newcommand{\projteam}{Grupo LP-07}

%% Package ---------------------------------------------------------------------
\usepackage[a4paper,left=25mm,right=25mm,top=25mm,bottom=25mm,headheight=6mm,footskip=12mm]{geometry}   % Document dimensions
\usepackage[T1]{fontenc}            % PS fonts
\usepackage[english]{babel}         % [portuges]??
\usepackage[export]{adjustbox}      %
\usepackage[normalem]{ulem}         % various types of underlining
\usepackage[table,xcdraw]{xcolor}   % driver-independent color extensions
\usepackage[utf8]{inputenc}         % accents
\usepackage{amsmath}                % multi-line and other mathematical statements
\usepackage{array}                  % Images in tables
\usepackage{booktabs}               %
\usepackage{caption}                % rotating captions, sideways captions, etc.
\usepackage{chicago}                % Bibliography style
\usepackage{color}                  %
\usepackage{fancyhdr}               % Headers and footers
\usepackage{float}                  % tables and figures in the multi-column environment 
\usepackage{graphicx}               % 
\usepackage{hyperref}               % Hyper references
\usepackage{lastpage}               % 
\usepackage{lipsum}                 % loren dummy text
\usepackage{listings}               % Programming syntax
\usepackage{longtable}              % Tables continue in the next page
\usepackage{multicol}               % 
\usepackage{multirow}               % tabular cells spanning multiple rows
\usepackage{newtxtext,newtxmath}    % do not use CM fonts
\usepackage{paralist}
\usepackage{setspace}               % setting the spacing between lines
\usepackage{subcaption}             % for subfigures and the like

%% Package settings ------------------------------------------------------------
\graphicspath{{./images}}                           % {graphicx} - Images path % BY PHASE!!!
\selectlanguage{english}                            % {babel} - Language portuguese
\setlength{\columnsep}{3cm}                         % {multicol} - Column spacement
\definecolor{engineering}{rgb}{0.549,0.176,0.098}   % {color}
\definecolor{cloudwhite}{cmyk}{0,0,0,0.025}         % {color}
\definecolor{deepblue}{rgb}{0,0,0.5}                % {color}
\definecolor{deepred}{rgb}{0.6,0,0}                 % {color}
\definecolor{deepgreen}{rgb}{0,0.5,0}               % {color}
\setlength{\parindent}{0em}                         % {geometry}
\setlength{\parskip}{1ex}                           % {geometry}
\lstdefinestyle{pythoncode}                         % {listings} - Python syntax
{
    aboveskip=8mm,
    backgroundcolor=\color{cloudwhite},             
    basicstyle=\footnotesize\ttfamily,
    numbers=left,                                   % where to put the line-numbers
    belowskip=4mm,
    breakatwhitespace=false,                        % sets if automatic breaks should only happen at whitespace
    breaklines=true,                                % sets automatic line breaking
    captionpos=b,                                   % sets the caption-position to bottom
    escapeinside={\%*}{*)},                         % if you want to add a comment within your code
    float=htb,
    frame=tb,
    keepspaces=true,
    keywordstyle=\bfseries\color{deepblue},
    morekeywords={*,var,template,new}               % if you want to add more keywords to the set
    numbersep=8pt,                                  % how far the line-numbers are from the code
    numberstyle=\scriptsize\texttt,                 % the size of the fonts that are used for the line-numbers
    showspaces=false,                               % show spaces adding particular underscores
    showstringspaces=false,                         % underline spaces within strings
    showtabs=false,                                 % show tabs within strings adding particular underscores
    stepnumber=1,                                   % the step between two line-numbers. If it's 1 each line will be numbered
    stringstyle=\color{deepgreen},
    tabsize=2,                                      % sets default tabsize to 2 spaces
}
\lstdefinestyle{termoutputs}                        % {listings} - Python syntax
{
    aboveskip=8mm,
    backgroundcolor=\color{cloudwhite},             
    basicstyle=\scriptsize\ttfamily,
    numbers=left,                                   % where to put the line-numbers
    belowskip=4mm,
    breakatwhitespace=false,                        % sets if automatic breaks should only happen at whitespace
    breaklines=true,                                % sets automatic line breaking
    captionpos=b,                                   % sets the caption-position to bottom
    escapeinside={\%*}{*)},                         % if you want to add a comment within your code
    float=htb,
    frame=tb,
    keepspaces=true,
    keywordstyle=\bfseries\color{deepblue},
    morekeywords={*,var,template,new}               % if you want to add more keywords to the set
    numbersep=8pt,                                  % how far the line-numbers are from the code
    numberstyle=\scriptsize\texttt,                 % the size of the fonts that are used for the line-numbers
    showspaces=false,                               % show spaces adding particular underscores
    showstringspaces=false,                         % underline spaces within strings
    showtabs=false,                                 % show tabs within strings adding particular underscores
    stepnumber=1,                                   % the step between two line-numbers. If it's 1 each line will be numbered
    stringstyle=\color{deepgreen},
    tabsize=2,                                      % sets default tabsize to 2 spaces
}
\fancyhf{}                                          % {fancyhdr} clear off all default fancyhdr headers and footers
\lfoot{\small{\emph{\projtitle, \projsubtitle}}}    % {fancyhdr}
\rfoot{\small{\thepage\ / \pageref{LastPage}}}      % {fancyhdr}
\pagestyle{fancy}                                   % {fancyhdr} apply the fancy header style
\renewcommand{\headrulewidth}{0.0pt}                % {fancyhdr} no head rule
\renewcommand{\footrulewidth}{0.4pt}                % {fancyhdr}
\hypersetup{                                        % {hyperref}
    plainpages=false,
    pdfpagelayout=SinglePage,
    bookmarksopen=false,
    bookmarksnumbered=true,
    breaklinks=true,
    linktocpage,
    colorlinks=true,
    linkcolor=engineering,
    urlcolor=engineering,
    filecolor=engineering,
    citecolor=engineering,
    allcolors=engineering
}

%% Document start --------------------------------------------------------------
\begin{document}
    \pagenumbering{roman}\setcounter{page}{1}

%% Cover -----------------------------------------------------------------------
\begin{titlepage}
    \center

    \vspace*{-12mm}
    {\large \textbf{\textsc{\school}}}\\

    \vfill

    \includegraphics[width=62mm]{logoisel}
    
    \vfill
    
    {\huge \textbf{\projtitle}}\\[6mm]
    {\Large \textbf{\projsubtitle}}\\
    
    \vfill
    
    \vfill
    
    {\Large \textbf{\projisel}}\\[12mm]
    
    {\Large \textbf{Coordination}}\\[4mm]
    {\large General: Carlos Meneses\hspace*{18mm}
            Course: Nuno Cruz}\\[6mm]
    
    {\Large \textbf{\projteam}}\\[4mm]
    {\large Supervisor: Luís Pires\hspace*{12mm}}\\[6mm]
    
    {\Large \textbf{Student}}\\[4mm]
    {\large Nuno Brito $<$A46948@alunos.isel.pt$>$}
    
    \vspace*{10mm}
    
    \renewcommand{\today}{June 2th 2024}
    \today
    
\end{titlepage}

%% TOC -------------------------------------------------------------------------
\tableofcontents

%% List of figures -------------------------------------------------------------
\listoffigures
\addcontentsline{toc}{chapter}{Figure list}

%% List of tables --------------------------------------------------------------
\listoftables
\addcontentsline{toc}{chapter}{Table list}

%% List of listings --------------------------------------------------------------
\lstlistoflistings
\addcontentsline{toc}{chapter}{Listings list}

%% Acronyms --------------------------------------------------------------------
\chapter*{Acronyms list}
    \addcontentsline{toc}{chapter}{Acronyms list}

    \begin{flushleft}
        \begin{tabular}{l p{0.8\linewidth}}
            API     & Application Programming Interface\\
            CLI     & Command Line Interface\\
            CMD     & Command Prompt\\
            GUI     & Graphical User Interface\\
            HTTP    & Hyper Text Transfer Protocol\\
            HTTPS   & Hyper Text Transfer Protocol Secure\\
            IP      & Internet Protocol\\
            IPv4    & Internet Protocol version 4\\
            IPv6    & Internet Protocol version 6\\
            LAN     & Local Area Network\\
            OS      & Operating System\\
            OSS     & openSUSE\\
            PC      & Personal Computer\\
            PHP     & PHP: Hypertext Preprocessor\\
            SSL     & Secure Sockets Layer\\
            TCP     & Transmission Control Protocol\\
            TLS     & Transport Layer Security\\
            TUI     & Terminal User Interface\\ % Not used yet
            UDP     & User Datagram Protocol\\
            VPN     & Virtual Private Network\\
            WWW     & World Wide Web\\
            XAMPP   & Cross-Platform, Apache, MySQL, PHP, and Perl
        \end{tabular}
    \end{flushleft}

%% Glossary --------------------------------------------------------------------
\chapter*{Glossary}
    \addcontentsline{toc}{chapter}{Glossary}

    \begin{description}
        \item[Apache2] \hfill \\
            An opensource HTTP web server.
        \item[Bit] \hfill \\
            A unit of information in computing and digital communications. The bit represents a logical state with one of two possible values, 0 or 1 (other representations such as \textit{true / false} are also valid).
        \item[Byte] \hfill \\
            Also a unit of digital information, consists of 8 bits.
        \item[Broadcast] \hfill \\
            A method of transferring a message to all recipients simultaneously.
        \item[Browser] \hfill \\
            A browser is a internet navigation software. It comes in multiple flavours, nowadays the big three are Microsoft Edge, Mozilla Firefox and Google Chrome.
        \item[Cisco Packet Tracer] \hfill \\
            A cross-platform visual network simulation tool.
        \item[Command Prompt] \hfill \\
            The default command-line interpreter for Windows operating systems.
        \item[Firewall] \hfill \\
            A barrier between networks. Controls inbound and outbound traffic.
        \item[Gateway] \hfill \\
            A network gateway provides a connection between networks and devices. Known as protocol translation gateways or mapping gateways, can perform protocol conversions to connect networks with different network protocol technologies.
        \item[LibreWolf] \hfill \\
            An internet browser based on Mozilla's Firefox. It's primary purpose is to allow privacy, and with it comes security. It achieves this by removing telemetry and data collection.
        \item[Linux] \hfill \\
            Open-source Unix-like operating systems based on the Linux kernel.
        \item[MariaDB] \hfill \\
            A community-developed fork of MySQL database server.
        \item[openSUSE Tumbleweed] \hfill \\
            An openSUSE (OSS) is an open-source community driven Linux-based distribuition sponsored by SUSE Software Solutions. Tumbleweed is a rolling release version allowing for up-to-date software releases.
        \item[Operating system] \hfill \\
            A program that manages a computer's resources from software to hardware.
        \item[Ping] \hfill \\
             A software utility used to test the reachability of a host on an IP network.
        \item[Tracert] \hfill \\
            Or \textbf{traceroute} in unix and linux systems, is a computer network diagnostic command for displaying possible routes and measuring transit delays of packets across an IP network.
        \item[Ipconfig] \hfill \\
            Or \textbf{ifconfig} in unix and linux systems, is a console application program that displays all current TCP/IP network configuration values.
        \item[Python] \hfill \\
            Python is a high-level programming language, object-oriented.
        \item[Perl] \hfill \\
            A high-level, general-purpose, interpreted, dynamic programming language
        \item[Rolling release distribuition] \hfill \\
            A distribuition where it's software release cycle is more frequent than those of Long Term Support (LTS). It's up to the Linux-based distribuitor to guarantee the testing of a package.
        \item[Router] \hfill \\
            A networking device that forwards data packets between computer networks, including internetworks such as the global Internet.
        \item[Switch] \hfill \\
            A networking hardware that connects devices on a computer network by using packet switching to receive and forward data to the destination device.
        \item[Socket] \hfill \\
            A network socket serves as an endpoint for sending and receiving data across the network.
        \item[Subnet Mask] \hfill \\
            Is a logical subdivision of an IP network.
        \item[Unix] \hfill \\
            Is a family of multitasking, multi-user computer operating systems that derive from the original AT\&T Unix.
        \item[VPN] \hfill \\
            A private network creating a secure connection between a device and a network.
        \item[Windows] \hfill \\
            Microsoft's operating system. First released in 1985 as a Graphical User Interface (GUI) for MS-DOS, continued to evolve with it's latest version being 11.
            Due to it's nature, it's not recommended for server production environment.
        \item[Wireshark] \hfill \\
            Wireshark is a network protocol analyser software. Allows traffic capture between a computer and a network.
        \item[XAMPP] \hfill \\
            A software package environment collection containing Apache2 webserver, MariaDB database, PHP and Perl.
    \end{description}

%% Chapter: introduction -------------------------------------------------------
\chapter{Introduction}
% display headers & footers
    \pagestyle{fancy}
    *placeover_text*

% main page numbers with arabic numerals
    \pagenumbering{arabic}\setcounter{page}{1}

%% Chapter: phase 4 ------------------------------------------------------------
\chapter{Phase 4}
    nota falar do servidor web
    dns / dhcp funcionamento (udp) e records
    introdução de relays 
    mencionar rotas (fase 3)
    correr NSLookup
    comandos cisco ip helper-address 192.168.7.1
    ou como remover usando no ip helper-address 192.168.7.x
    print com show ip interface FastEthernet0/0
    ipconfig nos equipamentos
    ipconfig /release /renew
    dns pools configuradas
    dns records configurados
    outputs e printscreens dos comandos e página web
    explicar o A record no dns server

    *placeover_text*
    \section{ChangeME}
    \section{Command line outputs}
        \lstset{style=termoutputs}
        \lstinputlisting[
            language={},
            caption={PC0 output (LAN A)},
            label={lst:pc0output}
        ]{./outputs/PC0.txt}

%% Chapter: issues and fixes ---------------------------------------------------
\chapter{Issues and fixes}
    \textbf{Cisco Packet Tracer in MacOS:}\\
        \hspace*{10mm}*STILL* no solution was found to deal with those annoying popups that takes primary focus over other windows, even using the latest version.\\

%% Chapter: conclusions --------------------------------------------------------
\chapter{Conclusions}
    *placeover_text*

%% Bibliography ----------------------------------------------------------------
%\renewcommand{\bibname}{Bibliographic references}
%\bibliographystyle{chicago}
%\bibliography{refs}
%\addcontentsline{toc}{chapter}{\refname}  % add it to table of contents

%% Appendix --------------------------------------------------------------------
\appendix
    \chapter{Outputs}
    \label{appendixref}
        \section{Command line encore}
        \label{clioutapxref}
            \lstset{style=termoutputs}
            \lstinputlisting[
                language={},
                caption={Laptop0 output},
                label={lst:laptop0output}
            ]{./outputs/Laptop0.txt}

\end{document}
